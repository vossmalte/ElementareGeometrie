\begin{defi}[Fläche]
Eine \emph{Fläche} ist eine glatte 2-dimensionale Mf.
\end{defi}


\begin{defi}[Orientierbarkeit]
Eine Fläche heißt \emph{nicht orientierbar}, wenn sie eine Teilmenge enthält, die zum Möbiusband homöomorph ist.
\end{defi}

\begin{bsp}[für nicht orientierbar]
Möbiusband; $\mathbb{RP}^2$ = \{ Geraden durch $0 \in \RR^3$ \}.
\end{bsp}

\begin{bem}
	Um neue Flächen zu gewinnen kann man Polygone mit gerader Kantenzahl nehmen und 
	paarweise kanten identifizieren.
\end{bem}

\begin{bem}[Systematisches Vorgehen]
	Gehe gegen den Uhrzeigersinn die Seiten des Polygons ab und gebe jeder Seite
	ein Symbol, e.g. $a$, falls a in Durchlaufrichtung zeigt, sonst $a^-1$.
\end{bem}

\begin{bsp}
	$S^2 = aa^{-1}$, 
	Donut $T^2 = aba^{-1}b^{-1}$, 
	$\RR\mathbb{P}^2 = abab = aa$,
	Kleinsche Flasche $K^2 = aba^{-1}b^{-1}$
\end{bsp}

\begin{defi}[Zusammenhängende Summe]
	Schneide aus $F$ und $F'$ eine abgeschlossene Kreisscheibe aus, 
	identifiziere die Ränder der Scheiben mit einem Homöomorphismus. 
	$\Rightarrow$ Zusammenhängende Summe $F\#F'$.
\end{defi}

\begin{bem}
	$\#$ ist kommutativ, assoziativ und das neutrale Element ist $S^2$. 
	$(\{F\}, \#)$ ist also eine Halbgruppe.
\end{bem}

\begin{bem}
	Entspricht $F$ dem Zeichensatz $A$ und $F'$ dem Zeichensatz $B$,
	so $F\#F'$ dem Zeichensatz $AB$.
\end{bem}

\begin{bsp}
	$a_1 b_1 a_1^{-1} b_1^{-1}\dots a_g b_g a_g^{-1} b_g^{-1}$
	entspricht einer geschlossenen orientierbaren Fläche mit $g$ Löchern.
\end{bsp}

\begin{satz}[Klassifikationssatz für Flächen]
Eine geschlossene Fläche ist entweder homöomorph zu
\begin{itemize}
	\item $S^2$ oder
	\item einer zusammenhängenden Summe von $T^2$ (Tori) oder
	\item einer zusammenhängenden Summe von $\mathbb{RP}^2$.
\end{itemize}
\end{satz}

\begin{lemma}
$\mathbb{RP}^2 \# \mathbb{RP}^2 \approx K^2$ (Kleinsche Flasche).
\end{lemma}

\begin{lemma}
$3 \mathbb{RP}^2 \approx T^2 \# \mathbb{RP}^2 $ (Kleinsche Flasche).
\end{lemma}

\begin{defi}[Simplex]
	$k\le n$, $v_0,\dots,v_k \in \RR^n$ affin unabhängig erzeugen den
	$k$-Simplex $\sigma = \sigma^k$ als konvexe Hülle der Punkte $v_0,\dots,v_k$.\\
	Also $\sigma = [v_0,\dots,v_k] = \{x\in\RR^n | x=\sum_{i=0}^n a_i v_i$ 
	mit $a_i\ge 0$ und $\sum_{i=0}^n a_i = 1\}$.
\end{defi}

\begin{bem}[Dimension, Seiten, Standardsimplex]~
	\begin{itemize}
		\item $k$ heißt die Dimension von $\sigma^k$.
		\item konvexe Hülle von einer Teilmenge an $v_i$ heißt Seite.
		\item Seiten sind auch immer Simplizes.
		\item 0-dim Seite: Ecke, 1-dim Seite: Kante von $\sigma$.
		\item $(k-1)$-dim Seite von $\sigma^k$: Seitenfläche, Seiten von Dimension $<k$: echte Seiten
		\item Standardsimplex aufgespannt von $[0, e_1,\dots, e_n] =: \Delta^k$
		\item Simplizes tragen Teilraum-Topologie, innere Punkte liegen \emph{nicht} auf einer echten Seite
	\end{itemize}
\end{bem}

\begin{bsp}[Simplizes]
	Strecke, Dreieck, Punkt.\\
	keine Simplizes: Strecke mit Punkt drauf, Viereck
\end{bsp}

\begin{bem}[Baryzentrische Koordinaten]
	$x=\sum_{i=0}^n a_i v_i$, dann heißt $(a_0,\dots,a_k)$ baryzentrische Koordinaten von x. 
	Baryzentrum = Schwerpunkt, $a_i$ Masse auf $v_i$.\\
	Baryzentrische Unterteilung in $(k+1)!$ kleinere Simplizes.
\end{bem}

\begin{defi}[Simplizialkomplex]
	ist eine Menge $K = \{\sigma_\lambda | \lambda \in \Lambda\}$ von Simplizes in $\RR^n$, wobei:
	\begin{enumerate}
		\item Die Vereinigung der Simplizes ist lokal endlich, d.h. jeder Punkt in 
		$|K| := \cup\sigma_\lambda$ besitzt Umgebung, die nur endlich viele der 
		$\sigma_\lambda$ trifft.
		\item Mit $\sigma_\lambda$ enthält $K$ auch alle Seiten von $\sigma_\lambda$.
		\item $\sigma_{\lambda_1} \cap \sigma_{\lambda_2}$ ist entweder leer oder 
		eine gemeinsame Seite beider Simplizes.
	\end{enumerate}
	Für $\{\sigma_\lambda | \lambda \in \Lambda\}$ endlich heißt $K$ endlicher Simplizialkomplex.
	$|K| in \RR^n$ (mit Teilraum-Topologie) heißt der $K$ zugrunde liegende topologische Raum
	oder zu $K$ gehörige Polyeder;\\
	$\dim K = \max \{\dim\sigma_\lambda | \lambda \in \Lambda\}$;\\
	$K'\subset K$ Teilkomplex als Vereinigung ausgewählter $\sigma_\lambda$.
\end{defi}

\begin{defi}[Triangulierbar]
	Ein topologischer Raum $X$ ist triangulierbar (Polyeder), wenn er Isomorph zu 
	einem Simplizialkomplex $K$ (zu dessen $|K|$) ist. 
	$K$ ist dann Triangulierung von $X$.
\end{defi}

\begin{bsp}
	$|K|$ ist natürlich triangulierbar mit Homöomorphismus $id_{|K|}$.\\
	Tetraeder ist homöomorph zur $S^2$.
\end{bsp}

\begin{bem}
	Ein Raum kann mehrere Triangulierungen besitzen.
\end{bem}

\begin{bem}
	Mit dem ebenen Polygonmodell sieht man, dass jede geschlossene Fläche triangulierbar ist,
	wenn man sie mit dem richtigen Netz aus 0-, 1-, und 2-Simplizes überdeckt.
\end{bem}

\begin{bem}
	Differenzierbare Mf sind immer triangulierbar, topologische Mf i.A. nur bis Dim 3.
	Dort tragen sie eine eindeutig bestimmbare differenzierbare Struktur.
\end{bem}

\begin{satz}[Euler'scher Polyerder-Satz]
	In konvexen Polyedern $P$ gilt stets $\chi := v-e+f = 2$, 
	wenn $P$ $v$ Ecken, $e$ Kanten und $f$ Seiten hat.
\end{satz}

\begin{defi}[Euler-Charakteristik]
	Sei $S$ eine geschlossene Fläche und 
	$\mathcal{T} = \{ T_1,\dots,T_f \} $ 
	eine Triangulierung von $S$ durch $f$ Dreiecke (2-Simplizes).
	Mit $v$ als Anzahl der Ecken und 
	$e$ die Anzahl der Kanten von $\mathcal{T}$ ist
	$\chi(S,\mathcal{T}) := v-e+f$ die Euler-Charakteristik 
	von $S$ bezüglich $\mathcal{T}$.
\end{defi}

\begin{satz}
	$\chi(S,\mathcal{T})$ ist unabhängig von $\mathcal{T}$ 
	und sogar unabhängig von jeder anderen polygonalen Zerlegung.
\end{satz}

\begin{bem}
	$\chi(S^2)=2$, 
	$\chi(T^2)=0$, 
	$\chi(\RR\mathbb{P}^2)=1$
\end{bem}

\begin{satz}
	Für geschlossene Flächen $S_1, S_2$ gilt:
	$\chi(S_1\#S_2) = \chi(S_1) + \chi(S_2) - 2$
\end{satz}

\begin{satz}
	$F:S_1\to S_2$ Homöomorphismus und 
	$\mathcal{T}$ Triangulierung von $S_1$,
	dann ist $F(\mathcal{T})$ 
	Triangulierung von $S_2$.
\end{satz}

\begin{satz}
	Sind zwei geschlossene Flächen homöomorph,
	so besitzen sie die gleiche Euler-Charakteristik.
\end{satz}

\begin{bem}
	Umkehrung gilt nicht, bspw:
	$\chi(K^2)=\chi(\RR\mathbb{P}^2\#\RR\mathbb{P}^2) = 2-2 = 0 = \chi(T^2)$
\end{bem}

\begin{satz}
	Zwei geschlossene Flächen sind genau dann homöomorph,
	wenn sie beide (nicht) orientierbar sind und
	gleiche Euler-Charakteristik besitzen.
\end{satz}

\begin{defi}[Genus]
	Das Geschlecht (Genus) einer geschlossenen Fläche ist 
	$g(S) = 0.5\times(2-\chi(S))$ für S orientierbar, sonst 
	$g(S) = 2-\chi(S)$.
\end{defi}

\begin{bem}
	Im endlichen Fall ist $g(S)$ einfach 
	die Anzahl der Löcher in $S$:
	$g(S^2)=0$, $g(T^2)=1$, $g(T^2\#T^2)=2$
\end{bem}