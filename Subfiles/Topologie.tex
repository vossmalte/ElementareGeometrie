\begin{defi}[Topologie]
  Sei $(X,\mathcal{O})$ mit $\mathcal{O} \subset 2^X$.
  $U \in \mathcal{O}$ nennt man offen.
  $\mathcal{O}$ heißt Topologie auf $X$, falls gelten:
  \begin{enumerate}
    \item $X, \emptyset \in \mathcal{O}$
    \item Beliebige Vereinigungen offener Mengen sind offen
    \item Endliche Schnitte offener Mengen sind offen
  \end{enumerate}
\end{defi}

\begin{defi}[Stetigkeit]
  $f:X\rightarrow Y$ heißt stetig, wenn die Urbilder offener Mengen stets offen sind.
\end{defi}

\begin{defi}[Abgeschlossen]
  $A \subset X$ abgeschlossen $:\Leftrightarrow X\backslash A$ offen
\end{defi}
\begin{defi}[Umgebung]
  $U \subset X$ heißt Umgebung von $x\in X$, falls
  $\exists V\in \mathcal{O}$ mit $x\in V \subset U$
\end{defi}
\begin{defi}[Innen, außen, Rand, Abschluss]
  ...
\end{defi}

\begin{defi}[Topologie eines metrischen Raums]
  Jeder metrische Raum ist ein topologischer Raum.
  Die induzierte Topologie $\mathcal{O}(d)$ geht so: $V$ heißt offen,
  wenn für jedes $x\in V$ ein $\varepsilon$-Ball um $x$ in $V$ liegt.
\end{defi}
\begin{defi}[$\mathcal{O}$ metrisierbar]
  $:\Leftrightarrow \exists d: \mathcal{O} = \mathcal{O}(d)$
\end{defi}

\begin{defi}[fein, grob]
  $\mathcal{O} \subset \mathcal{O}' \Leftrightarrow$
  $\mathcal{O}'$ feiner als $\mathcal{O}$
\end{defi}

\begin{defi}[disjunkte Summe / Vereinigung]
  von $X, Y$ erklärt als
  $X+Y = X\times Y:= X\times {0} \cup Y\times {1}$
\end{defi}

\begin{defi}[Topologische Summe]
  $(X,\mathcal{O}), (Y,\mathcal(O)')$ top. Räume.
  $\{U+V | U \in \mathcal{O}, V \in \mathcal{O}'\}$
  ist Topologie auf $X+Y$. $X+Y$ dann top. Summe.
\end{defi}

\begin{defi}[Produkttopologie]
  $W \subset X\times Y$ offen
  $:\Leftrightarrow \forall (x,y) \in W: \exists$
  Umgebung $U$ von $x$ in X und Umgebung $V$ von $y$ in Y
  mit $U \times V \subset W$.
\end{defi}

\begin{bem}
  Solche \glqq Rechtecke\grqq{} sind offen in der Produkttopologie,
  aber nicht alle offenen Mengen sind solche Rechtecke.
\end{bem}

\begin{defi}[Basis]
  Menge $\mathcal{B}$ Basis $:\Leftrightarrow$ jede offene Menge
  in $X$ ist Vereinigung von solchen aus $\mathcal{B}$
\end{defi}

\begin{defi}[Subbasis]
  Menge $\mathcal{S}$ Subbasis $:\Leftrightarrow$ jede offene Menge
  in $X$ ist Vereinigung von endlich viele Durchschnitten
  von Mengen aus $\mathcal{B}$\\
  $\Leftrightarrow$ endliche Schnitte in $\mathcal{S}$ sind Basis.
\end{defi}

\begin{bem}
  Für $S \subset 2^X$ gibt es genau eine Topologie für die $S$ Subbasis ist.
\end{bem}

\begin{bem}
  \begin{itemize}
    \item Kompositionen stetiger Abbildungen sind stetig
    \item $f:X\to Y$ stetig, $X_0 \subset X \Rightarrow f|_{X_0}$ stetig
    \item $f:X+Y\to Z$ stetig $\Leftrightarrow f|_X, f|_Y$ stetig
    \item $(f,g): Z\to X\times Y$ stetig $\Leftrightarrow f:Z\to X, g:Z\to Y$ stetig
  \end{itemize}
\end{bem}

\begin{defi}[Homömorphismus]
  $f:X\to Y$ heißt Homömorphismus, wenn $f$ bijektiv, $f$ stetig, $f^{-1}$ stetig.\\
  X und Y sind dann homöomorph ($X\approx Y$).
\end{defi}

\begin{bem}
  Homömorphismen sind strukturerhaltende (offen, abgeschlossen, Umgebung, (Sub-)Basis) 
  Abbildungen zwischen top. Räumen.
\end{bem}

\begin{defi}[Zusammenhang]
  X heißt zusammenhängend, wenn er sich nicht in zwei nichtleere, offene und
  disjunkte Teilmengen zerlegen lässt.\\
  $\Leftrightarrow X, \emptyset$ sind die einzigen Teilmengen, die offen und abgeschlossen sind.
\end{defi}

\begin{bem}
  Zusammenhang ist eine Invariante, also $X, Y$ nicht homöomorph, 
  wenn einer zusammenhängend, der andere nicht.
\end{bem}

$\dots$

\begin{defi}[Wegzusammenhang]
Ein topologischer Raum $X$ heißt \emph{wegzusammenhängend}, 
wenn es zu je zwei Punkten $a, b \in X$ 
einen Weg von $a$ nach $b$ gibt, 
d.h. eine stetige Abbildung $f: [0, 1] \to X$ mit $f(0) = a$, $f(1) = b$.
\end{defi}

\begin{bem} Es gilt:
\begin{itemize}
\item Wegzusammenhängende Räume sind zusammenhängend.
\item Nichtdisjunkte Vereinigungen von (weg-)zusammenhängenden Räumen sind (weg-)zusammenhängend
\item $X \times Y$ wegzusammenhängend $\Leftrightarrow$ $X$ und $Y$ jeweils wegzusammenhängend
\item Stetige Bilder von (weg-)zusammenhängenden Mengen sind (weg-)zusammenhängend. \\
\end{itemize}
\end{bem}


\begin{defi}[Hausdorffsch]
Ein topologischer Raum $X$ heißt \emph{hausdorffsch}, wenn es zu je zwei verschiedenen Punkten $x, y \in X$ disjunkte Umgebungen von $x$ und $y$ gibt.
\end{defi}

\begin{bem} Jeder metrische Raum ist hausdorffsch. \\
\end{bem}

\begin{defi}[Kompaktheit]
Eine Menge / topologischer Raum $X$ heißt \emph{kompakt}, wenn jede offene Überdeckung von $X$ eine endliche Teilüberdeckung besitzt.
\end{defi}

\begin{bem} $X \neq \emptyset \neq Y$ sind kompakt $\Leftrightarrow$ $X + Y$ kompakt $\Leftrightarrow$ $X \times Y$ kompakt.
\end{bem}

\begin{bsp} Es gilt:
\begin{itemize}
\item Auf der diskreten Topologie gilt: $X$ kompakt $\Leftrightarrow$ $X$ endlich.
\item Abgeschlossene Intervalle in $\RR$ sind kompakt.
\item Alle abgeschlossenen und beschränkten Teilmengen des $\RR^N$ sind kompakt.
\item Abgeschlossene Teilmengen von einem Kompaktum sind kompakt.
\end{itemize}
\end{bsp}

\begin{bem}
Kompaktheit erlaubt \emph{Schlüsse von lokalen auf globale Eigenschaften}. Bspw.:
\begin{itemize}
	\item Ist $f: X \to \RR$ (stetig) lokal beschränkt, so auch global.
	\item Ist $\{A_i\}$ eine lokal endliche Überdeckung von X, so ist die Überdeckung endlich. \\
\end{itemize}
\end{bem}

\begin{satz}[von Heine-Borel] Die kompakten Teilmengen des $\RR^n$ sind genau die abgeschlossenen und beschränkten.
\end{satz}

\begin{kor}
Stetige Funktionen auf Kompakta sind beschränkt.
\end{kor}
\begin{kor}
$X$ Hausdorff-Raum, $K \subseteq X$ kompakt, so $K$ abgeschlossen in $X$. \\
\end{kor}


\begin{defi}[Limes]
$(x_n)$ Folge in $X$. Dann ist $a \in X$ ist \emph{Limes} der Folge, wenn sich in jeder Umgebung U von $a$ fast alle Folgenglieder befinden.
\end{defi}
\begin{bem}
In Hausdorff-Räumen sind Limiten eindeutig.
\end{bem}
