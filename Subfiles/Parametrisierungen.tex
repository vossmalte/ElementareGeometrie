\begin{defi}
    Eine Fläche im $\RR^3$ ist eine glatte Untermannigfaltigkei
    der Dimension 2 des $\RR^3$.
\end{defi}

\begin{bem}
    $F\subset \RR^3$. Zu jedem Punkt von F gibt es eine Umgebung $U$ und eine Abbildung
    $r:V\to\RR^3$ von einer offenen Teilmenge $V\subset\RR^2$, sodass
    \begin{enumerate}
        \item $r:V\to U$ ist Homöomorphismus.
        \item $r(u,v) = (x(u,v),y(u,v),z(u,v))$ mit $x,y,z:V\to \RR$ ist glatt.
        \item $r_u := \frac{\partial r}{\partial u} = 
        (\frac{\partial x}{\partial u},\frac{\partial y}{\partial u}, \frac{\partial z}{\partial u}) = 
        r_x\frac{\partial}{\partial u} = Dr\cdot\frac{\partial }{\partial u}$
        und $r_v$ sind linear unabhängig.
    \end{enumerate}
\end{bem}

\begin{defi}{Lokale Parametrisierung}
    $r:V\subset \RR^2\to\RR^3$ mit $r(V)\subset F$.
    Dabei gelten 2) und 3), 1) nicht notwending.
\end{defi}

\begin{bsp}{Parametrisierungen}
    \begin{itemize}
        \item Torus $T^2$: $r(u,v) = (a+b\cos u)(\cos v \cdot e_1 + \sin v \cdot e_2) + b\sin u \cdot e_3$
        \item Sphäre $S^2$: $r(u,v) = a\sin u \sin v \cdot e_1 + a\cos u \sin v \cdot e_2 + a\cos v \cdot e_3$
        \item Rotationsfläche: $r(u,v) = f(u)(\cos v \cdot e_1 + \sin v \cdot e_2) + u \cdot e_3$
        \item Zylinder: $r(u,v) = a(\cos v \cdot e_1 + \sin v \cdot e_2) + u \cdot e_3$
        \item Kegel: $r(u,v) = au(\cos v \cdot e_1 + \sin v \cdot e_2) + u \dot e_3$
        \item Helikorid: $r(u,v) = a \cos v \cdot e_1 + au\sin v \cdot e_1 + v\cdot e_3$ (Wendelfläche)
    \end{itemize}
\end{bsp}

\begin{bem}
    Kartenwechsel entspricht Umparametrisierung.
\end{bem}

\begin{defi}{Einheitsnormalenvektoren}
    ... % TODO
\end{defi}

\begin{defi}{Glatte Kurve}
    in F ist eine Immersion
    $\gamma:\RR\supset I\to F$ 
    mit $\gamma'(t) \neq 0 \forall t$ (also nicht stehen bleiben).
\end{defi}

\begin{bem}
    $r$ Parametrisierung von $F$, so ist bzgl. $r$ eine glatte Kurve in F gegeben durch
    eine glatte Abbildung $t\mapsto (u(t),v(t))\subset V$, 
    $\gamma(t) = r(u(t), v(t)) \subset F$
\end{bem}

\begin{defi}{Erste Fundamentalform}
    $\gamma:[a,b] \to F$ glatte Kurve bzgl $r$. 

    $\ell(\gamma) = \int_a^b |\gamma'(t)|dt = 
        \int_a^b \sqrt{(r_u\cdot u'+r_v\cdot v')(r_u\cdot u+r_v\cdot v')} dt =
        \int_a^b \sqrt{E\cdot u'^2 + 2Fu'v' + Gv'^2} dt$

    mit $E=r_u\cdot r_u, F=r_u\cdot r_v, G=r_v\cdot r_v$.
    
    $I:= Edu^2 + 2Fdudv + Gdv^2$ heißt erste Fundamentalform.
\end{defi}

\begin{bem}
    $I$ ist nichts anderes als das euklidische Skalarprodukt eingeschränkt auf
    $T_aF$ aufgespannt von $r_u, r_v$.
    In dieser Basis gegeben durch 
    $\begin{psmallmatrix}
        E & F \\ F & G \\
    \end{psmallmatrix}$.
    Also $E>0, G>0, EG-F^2>0$!
\end{bem}

\begin{bem}{Änderung der Parametrisierung}
    $\begin{psmallmatrix}
        E' & F' \\ F' & G' \\
    \end{psmallmatrix} = 
    \begin{psmallmatrix}
        u_x & u_y \\ v_x & v_y \\
    \end{psmallmatrix}
    \cdot
    \begin{psmallmatrix}
        E & F \\ F & G \\
    \end{psmallmatrix}
    \cdot
    \begin{psmallmatrix}
        u_x & u_y \\ v_x & v_y \\
    \end{psmallmatrix}$
\end{bem}

\begin{bem}{Flächeninhalt eines Gebiets}
    $A(V):=\int_V|r_u\times r_v|dudv = \int_V\sqrt{EG-F^2}dudv$
\end{bem}

\begin{defi}{kurven-isometrisch}
    Zwei Flächen heißen kurven-isometrisch, falls es einen Diffeomorphismus gibt,
    der Kurven auf Kurven gleicher Länge abbildet: 
    $\ell(\gamma)=\ell(f\circ \gamma) \forall \gamma:[0,1]\to F$
\end{defi}

\begin{defi}{lokale Kurvenisometrie}
    $f:F\to F'$, falls $\exists$ offene Teilmenge $V\subset F$, sodass
    $f|_V:V\to f(V)$ eine Isometrie ist.
\end{defi}

\begin{satz}
    Zwei Kartenbereiche $U, U'$ von Flächen $F, F'$ im $\RR^3$ sind isometrisch genau dann, wenn
    $\exists$ Parametrisierungen $r:V\to U, r':V'\to U'$, deren erste Fundamentalform übereinstimmen.
\end{satz}

\begin{bsp}
    Kegel und Ebene sind lokal isometrisch.
\end{bsp}

\begin{defi}{Zweite Fundamentalform}
    $II:=-(r_u \cdot n_u \cdot du^2 + (r_u \cdot n_v + r_v \cdot n_u)dudv + r_v \cdot n_v \cdot dv^2)$
        $=\frac{1}{2}\frac{\partial}{\partial t}|_{t=0}I(t)$
        $=Ldu^2+2Mdudv+Ndv^2$
        mit $L=r_{uu}\cdot n, M=r_{uv}\cdot n, N=r_vv\cdot n$
\end{defi}

\begin{bsp}
    \begin{itemize}
        \item Ebene: $II=0$
        \item Sphäre mit Radius $a>0$: $II=\frac{1}{a}I$
    \end{itemize}
\end{bsp}

\begin{satz}
    Verschwindet die zweite Fundamentalform, so ist die Fläche ein Teil der Ebene.
\end{satz}

\begin{defi}{Gauß-Krümmung}
    ist die Funktion $K:=\frac{LN-M^2}{EG-F^2}=\frac{\det II}{\det I}$
\end{defi}

\begin{bsp}
    \begin{itemize}
        \item Ebene: $K=0$
        \item Sphäre: $K=\frac{1}{a^2}$
    \end{itemize}
\end{bsp}

\begin{satz}{Theorema Egregium}
    $\det II$ ist durch $I$ und ihre Ableitungen vollständig bestimmt.
\end{satz}

\begin{defi}{Geodätische Krümmung}
    $\kappa_g := t'\cdot n\times t$ mit $t:=\frac{\partial \gamma}{\partial s}$
    von einer parametrisierten Kurve $\gamma$
\end{defi}

\begin{defi}{Einfach geschlossene Kurve}
    ist eine glatte Kurve, deren Bild zu $S^1$ homöomorph ist.
\end{defi}

\begin{satz}{Gauß-Bonnet I}
    $\int_\gamma \kappa_g ds = 2\pi - \int_R KdA$,
    $\gamma$ umschließt $R$.
\end{satz}

\begin{kor}
    Winkelsumme $\alpha + \beta + \gamma = \pi + \int_R KdA + \int_\gamma \kappa_g ds$
\end{kor}

\begin{bsp}
    Ebene: $\kappa_g=0,K=0 \Rightarrow \alpha + \beta + \gamma = \pi$
\end{bsp}

\begin{satz}{Gauß-Bonnet II}
    $F$ glatte, geschlossene, orientierte Fläche, so gilt
    $\int_F KdA = 2\pi \cdot \chi(F)$
\end{satz}

\begin{bsp}
    \begin{itemize}
        \item $K(S^2)=1 \Rightarrow \chi(S^2)=2$
        \item $\chi(S^2)=2 \Rightarrow$ Für jede Metrik auf $S^2$ ist $\int_{S^2}KdA=4\pi>0 
            \Rightarrow es gibt keinen globalen planaren Atlas$
        \item $\chi(T^2)=0 \Rightarrow $ mittlere Krümmung 0
    \end{itemize}
\end{bsp}