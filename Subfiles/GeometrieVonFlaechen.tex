\begin{defi}[Abstand auf einer Fläche im $\RR^3$]
    Das Bogenlängeintegral $\mathcal{l}(\gamma):=\int||\gamma'(t)||dt$
    $=\int\sqrt{<\gamma'(t),\gamma'(t)>}$
\end{defi}

\begin{defi}[Abstand auf einer allgemeinen Fläche]
    Hier wird eine Riemann'sche Metrik auf $S$ (genauer: $TS$) benötigt.
    Karte von $TS$: $\Phi\times id: U\times\RR^2\to\Phi(U)\times\RR^2$,
    $(p,u\partial_x + v\partial_y) \mapsto (x,y,u,v)$ mit $\Phi(p)=(x,y)$.
\end{defi}

\begin{defi}[Skalarprodukt]
    Ein SP auf $T_pS$ ist eine Funktion $<\dot,\dot>_p: T_pS\times T_pS\to\RR$
    und es ist symmetrisch, bilinear und positiv definit.
\end{defi}

\begin{defi}[Riemann'sche Metrik]
    auf $S$ ist eine Familie von Skalarprodukten auf allen Tangentialräumen $T_pS$ von $S$,
    die glatt von $p$ abhängt.
\end{defi}

\begin{bem}
    Matrix $A$ beschreibt Riemann'sche Metrik, 
    transformiert bei Koordinatenwechsel in $C^tAC$
    mit $C$ Jacobi-Matrix des Kartenwechsels.
\end{bem}

\begin{bsp}[Polarkoordinaten]
    $x=r\sin(\phi)$, $y=r\cos(\phi)$, $u=u(x,y)$, dann
    $\frac{\partial u}{\partial r} = \frac{\partial u}{\partial x}\frac{\partial x}{\partial r} + $
    $\frac{\partial u}{\partial y}\frac{\partial y}{\partial r}$
    mit Kettenregel. 
    $\leadsto r\frac{\partial}{\partial r} = x\frac{\partial}{\partial x} + y\frac{\partial}{\partial y}$
\end{bsp}

\begin{bem}
    Lokal gibt es immer Riemann'sche Metrik auf einer Mannigfaltigkeit.
\end{bem}

\begin{defi}
    Für $u\in T_pS$ gilt:
    \begin{itemize}
        \item Länge (Norm): $||u||_p:=\sqrt{<u,u>_p}$
        \item Winkel: $\alpha=\arccos\frac{<u,v>_p}{||u||_p||v||_p}$
        \item Parallelogramm Fläche: $||u||_p||v||_p\sin\alpha$
    \end{itemize}
\end{defi}

% TODO: Winkel zwischen Kurven