\begin{defi}{Halbebene}
    $\mathbb{H}:= \{(x,y) \in \mathbb{R}^2 | y > 0\} = \{z \in \mathbb{C} | Im(z)>0\}$
    mit Metrik $ds^2 = \frac{1}{y^2}(dx^2+dy^2)$
\end{defi}

\begin{bem}
    Euklidische Abstände werden also 
    vergrößert,  wenn $y$ klein ist und 
    verkleinert, wenn $y$ groß ist.
\end{bem}

\begin{bem}
    Isometriegruppe $Isom(\mathbb{H})$ wirkt transitiv auf $\mathbb{H}$,
    also zu $p=p_1+ip_2,q=q_1+iq_2$ gibt es Isometrie $f$ mit $f(p)=q$.
    $p \mapsto \frac{q_2}{p_2}p + (q_1 - \frac{q_2}{p_2}p_1)$.
\end{bem}

\begin{defi}{Poincaré'sches Scheibenmodell}
    $ D = \{z \in \mathbb{C} | |z|<1\}$ 
    mit Metrik $ds^2 = 4\frac{dx^2+dy^2}{(1-x^2-y^2)^2}$
\end{defi}

\begin{bem}
    $D$ ist das Bild von $\mathbb{C}$ unter der konformen Abbildung 
    $z\mapsto\frac{iz+1}{z+i}$.
\end{bem}

\begin{bem}{Isometrien von $\mathbb{H}^2$}
    \begin{itemize}
        \item Translation in x-Richtung: $t \in \RR, z \mapsto z+t$
        \item Homothetien: $\lambda > 0, z \mapsto \lambda z$
        \item $z \mapsto -\frac{1}{z}$
        \item Möbiustransformation: $a,b,c,d \in \RR, ad-bc=1,z \mapsto \frac{az+b}{cz+d}$
        \item Isometriegruppe isomorph zu $SL_2(\RR)$
    \end{itemize}
    Alle Isometrien sind Möbiustransformationen!
\end{bem}

% TODO: Scheibenmodell

\begin{bem}{Geodätische}
    \begin{itemize}
        \item Abstand zwischen $iy_1, iy_2$: $\log(y_2)-\log(y_2)$ (Senkrechte)
        \item Vertikale Geradenstücke sind Geodätische
        \item $-\frac{1}{z}$ angewandt auf Vertikale Geraden $=$ Halbkreis mit Zentrum auf der reellen Achse
        \item also alle solchen Halbkreise auch Geodätische
    \end{itemize}
\end{bem}

\begin{defi}{Doppelverhältnis}
    $z_1,z_2,z_3,z_4 \in \mathbb{C}$, so ist das Doppelverhältnis
    $(z_1,z_2;z_3,z_4) := \frac{z_1-z_3}{z_2-z_3} / \frac{z_1-z_4}{z_2-z_4}$
\end{defi}

\begin{satz}
    Ist $\gamma$ eine Gerade oder ein Kreis in $\mathbb{C}$ und 
    $f$ eine gebrochen-lineare Transformation,
    so ist $f(\gamma)$ auch eine Gerade oder ein Kreis.
\end{satz}

\begin{bem}{Allgemeine Formel für $d(z_1,z_2)$}
    Mit Invarianz des Doppelverhältnisses:
    $d(z_1,z_2) = \log|\frac{z_1-w_1}{z_2-w_1}| - \log|\frac{z_1-w_2}{z_2-w_2}|$
\end{bem}
