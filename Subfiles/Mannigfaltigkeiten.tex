\begin{defi}[Topologische Mannigfaltigkeit]
Eine \emph{topologische Manngifaltigkeit} $M$ ist ein Hausdorff-Raum mit abzählbarer Basis der Topologie, sodass jeder Punkt $p$ aus $M$ eine offene Umgebung besitzt, die homöomorph zu einer offenen Menge von $\RR^N$ ist. ("die lokal so aussieht wie der $\RR^N$")
\end{defi}

\begin{defi}[Karte, Atlas]
Ein solcher Homöomorphismus $\phi: U \to V \subseteq \RR^N$ heißt \emph{Karte (um $p$)}; ein \emph{Atlas} von $M$ ist eine Menge $\mathcal{A}$ von Karten von M, deren Definitionsbereiche ganz $M$ überdecken.
\end{defi}

\begin{defi}[Kartenwechsel]
Für $(U, \phi)$ und $(V, \psi)$ aus $\mathcal{A}$ heißt \underline{$\psi \circ \phi^{-1}$} $: \phi(U \cap V) \to \psi(U \cap V)$ \emph{Kartenwechsel}.
\end{defi}

\begin{defi}[glatt, maximal, $C^\infty$]
Ein $C^{\infty}$-Atlas ist ein Atlas auf $M$, für den alle Kartenwechsel $C^{\infty}$-Diffeomorphismen sind, und er heißt \emph{maximal}, wenn er in keinem anderen echt enthalten ist.
In diesem Fall nennt man $\mathcal{A}$ auch \emph{differenzierbare} oder \emph{glatte Struktur} auf $M$ und nennt $M$ \emph{differenzierbare Mannigfaltigkeit}.
\end{defi}

\begin{bsp} Beispiele für glatte Mannigfaltigkeiten:
\begin{itemize}
	\item $\RR^N$ und alle offenen Teilmengen
	\item $S^1$
	\item Produkte $M \times N$ von glatten Mannigfaltigkeiten
	\item $T^N := S^1 \times \dots \times S^1$, der \emph{N-dimensionale Torus}
	\item GL$(n, \RR) = \det^{-1}(\RR \backslash \{0\})$
\end{itemize}	
\end{bsp}
