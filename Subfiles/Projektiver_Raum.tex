\begin{defi}[Projektiver Raum]
Wir definieren $\mathbb{RP}^N = P(\RR^{N+1})$, den \emph{n-dimensionalen reellen projektiven Raum}, als die Menge der Äquivalenzklassen von Elementen aus $\RR^{N+1}$ unter der Äquivalenzrelation $\sim$: $x \sim y \Leftrightarrow \exists \lambda \in \RR \backslash \{0\}: x = \lambda \cdot y.$
\end{defi}

\begin{satz}
Der $\mathbb{RP}^N$ ist eine N-dimensionale differenzierbare Mannigfaltigkeit.
\end{satz}

\begin{bem}
Der $\mathbb{CP}^N = P(\mathbb{C}^{N+1})$ wird mit der gleichen Äquivalenzrelation wie oben definiert, nur dass jetzt $\lambda \in \mathbb{C} \backslash \{0\}$.
Der $\mathbb{CP}^N$ ist eine \emph{reelle} (!) $2N$-dimensionale differenzierbare Mannigfaltigkeit.
\end{bem}

\begin{bem}[Isomorphie zur $S^N$] Es gilt:
\begin{itemize}
	\item $\mathbb{RP}^N \cong S^N / \sim$ mit $x \sim y \Leftrightarrow x = \pm y$.
	\item $\mathbb{CP}^N \cong S^{2N+1} / \sim$ mit $x \sim y \Leftrightarrow x = \lambda \cdot y$ für $\lambda \in S^1$.
\end{itemize}
\end{bem}