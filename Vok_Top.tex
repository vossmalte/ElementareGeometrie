$\dots$

\begin{defi}[Wegzusammenhang]
Ein topologischer Raum $X$ hei�t \emph{wegzusammenh�ngend}, wenn es zu je zwei Punkten $a, b \in X$ einen Weg von $a$ nach $b$ gibt, d.h. eine stetige Abbildung $f: [0, 1] \to X$ mit $f(0) = a$, $f(1) = b$.
\end{defi}

\begin{bem} Es gilt:
\begin{itemize}
\item Wegzusammenh�ngende R�ume sind zusammenh�ngend.
\item Nichtdisjunkte Vereinigungen von (weg-)zusammenh�ngenden R�umen sind (weg-)zusammenh�ngend
\item $X \times Y$ wegzusammenh�ngend $\Leftrightarrow$ $X$ und $Y$ jeweils wegzusammenh�ngend
\item Stetige Bilder von (weg-)zusammenh�ngenden Mengen sind (weg-)zusammenh�ngend. \\
\end{enumerate}
\end{bem}


\begin{defi}[Hausdorffsch]
Ein topologischer Raum $X$ hei�t \emph{hausdorffsch}, wenn es zu je zwei verschiedenen Punkten $x, y \in X$ disjunkte Umgebungen von $x$ und $y$ gibt.
\end{defi}

\begin{bem} Jeder metrische Raum ist hausdorffsch. \\
\end{bem}

\begin{defi}[Kompaktheit]
Eine Menge / topologischer Raum $X$ hei�t \emph{kompakt}, wenn jede offene �berdeckung von $X$ eine endliche Teil�berdeckung besitzt.
\end{defi}

\begin{bem} $X \neq \emptyset \neq Y$ sind kompakt $\Leftrightarrow$ $X + Y$ kompakt $\Leftrightarrow$ $X \times Y$ kompakt.
\end{bem}

\begin{bsp} Es gilt:
\begin{itemize}
\item Auf der diskreten Topologie gilt: $X$ kompakt $\Leftrightarrow$ $X$ endlich.
\item Abgeschlossene Intervalle in $\RR$ sind kompakt.
\item Alle abgeschlossenen und beschr�nkten Teilmengen des $\RR^N$ sind kompakt.
\item Abgeschlossene Teilmengen von einem Kompaktum sind kompakt. \\
\end{enumerate}
\end{bsp}

\begin{satz}[von Heine-Borel] Die kompakten Teilmengen des $\RR^n$ sind genau die abgeschlossenen und beschr�nkten.	
\end{satz}

\begin{kor}
Stetige Funktionen auf Kompakta sind beschr�nkt.	
\end{kor}
\begin{kor}
$X$ Hausdorff-Raum, $K \subseteq X$ kompakt, so $K$ abgeschlossen in $X$. \\
\end{kor}

\paragraph*{Schl�sse von lokalen auf globale Eigenschaften}

$\dots$